\documentclass{uebung}

\begin{document}
\maketitle{1}

\begin{exercise}[Urteilsgleichheit ist eine Kongruenzrelation]
  Formuliere Regeln, welche die folgenden Aussagen umsetzen:
  \begin{enumerate}
    \item Urteilsgleichheit von Typen ist eine Äquivalenzrelation zwischen Typen.
    \item Urteilsgleichheit von Termen ist eine Äquivalenzrelation zwischen Termen desselben Typs.
  \end{enumerate}
  Die folgende Regel drückt aus, dass Urteilsgleichheit von Typen mit der Formationsregel $\Pi\mathrm{F}$ verträglich ist.
  $$
  \inferrule{
    \Gamma\vdash A\equiv A'\text{ Typ} \\ \Gamma,x:A\vdash B(x)\equiv B'(x)\text{ Typ}
  }{
    \Gamma\vdash \Pi_{x:A}B(x)\equiv\Pi_{x:A'}B'(x)\text{ Typ}
  }{
    \Pi\mathrm{F}\equiv
  }
  $$
  Formuliere auf analoge Weise Regeln, welche die folgenden Aussagen umsetzen:
  \begin{enumerate}[start=3]
    \item Urteilsgleichheit von Termen ist verträglich mit der Einführungsregel $\Pi\mathrm{I}$.
    \item Urteilsgleichheit von Termen ist verträglich mit der Eliminationsregel $\Pi\mathrm{E}$.
  \end{enumerate}
\end{exercise}

\begin{exercise}[Extensionale Urteilsgleichheit abhängiger Funktionen]
  Seien $f:\Pi_{x:A}B(x)$ und $g:\Pi_{x:A}B(x)$ abhängige Funktionen, sodass $f(x)\equiv g(x):B(x)$ in einem Kontext $\Gamma,x:A$ gilt.
  Gib eine Herleitung für $f\equiv g:\Pi_{x:A}B(x)$ im Kontext $\Gamma$ an.
\end{exercise}

\begin{exercise}[Assoziativität der Komposition]
  Seien $f:A\to B$, $g:B\to C$ und $h:C\to D$ Funktionen in einem Kontext $\Gamma$.
  \begin{enumerate}
    \item Stelle einen Herleitungsbaum für $((h\circ g)\circ f)(x)\equiv h(g(f(x))):D$ im Kontext $\Gamma,x:A$ auf.
    \item Stelle einen Herleitungsbaum für $(h\circ (g\circ f))(x)\equiv h(g(f(x))):D$ im Kontext $\Gamma,x:A$ auf.
    \item Kombiniere dies zu einer Herleitung von $h\circ(g\circ f)\equiv (h\circ g)\circ f:A\to D$ im Kontext $\Gamma$.
  \end{enumerate}
\end{exercise}

\begin{exercise}[Konstante Funktionen]
  Sei $B$ ein Typ in einem Kontext $\Gamma$.
  \begin{enumerate}
    \item Definiere die konstante Funktion $\mathrm{const}_{B,C,z}:B\to C$ im Kontext $\Gamma,z:C$.
    \item Sei $f:A\to B$ eine Funktion im Kontext $\Gamma$.
      Zeige $\mathrm{const}_{B,C,z}\circ f \equiv \mathrm{const}_{A,C,z}:A\to C$ im Kontext $\Gamma,z:C$.
    \item Sei $g:C\to D$ eine Funktion im Kontext $\Gamma$.
      Zeige $g\circ\mathrm{const}_{B,C,z} \equiv \mathrm{const}_{B,D,g(z)}:B\to D$ im Kontext $\Gamma,z:C$.
  \end{enumerate}
\end{exercise}

\end{document}
