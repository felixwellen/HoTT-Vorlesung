\documentclass{uebung}

\begin{document}
\maketitle{11}

\begin{exercise}[Funktorialität der Verschleifung]
  Seien $(f,p_f):(A,a)\to (B,b)$ und $(g,p_g):(B,b)\to (C,c)$ punktierte Abbildungen.
  \begin{enumerate}
    \item Finde eine Gleichheit $p_{g\circ f}:(g\circ f)(a)=c$, sodass $(g\circ f,p_{g\circ f})$ eine punktierte Abbildung $(A,a)\to (C,c)$ wird.
    \item Zeige, dass $\Omega(g,p_g)\circ\Omega(f,p_f) \sim \Omega(g\circ f,p_{g\circ f})$ als Abbildungen $A\to C$.
    \item Mit Funktionsextensionalität sind die beiden Abbildungen sogar gleich.
      Sind sie aber auch gleich als punktierte Abbildungen?
  \end{enumerate}
\end{exercise}

\begin{exercise}[$\Omega f$ ist eine Vergleichsabbildung]
  Sei $(A,a)$ ein punktierter Typ.
  Laut Bemerkung 3.4.5 ist der Schleifenraum von $(A,a)$ die Spitze eines Pullback-Quadrates:
  \begin{center}
    \begin{tikzcd}
      \Omega(A,a)
      \arrow[r]
      \arrow[d]
      & \einheit
      \arrow[d, "a"]
      \\
      \einheit
      \arrow[r, "a"]
      & A
    \end{tikzcd}
  \end{center}
  Hier fassen wir den Basispunkt $a:A$ als die Abbildung $\rec\einheit(A,a):\einheit \to A$ auf.
  Sei nun $(f,p_f):(A,a)\to (B,b)$ eine punktierte Abbildung.
  Finde ein kommutatives Quadrat der Form
  \begin{center}
    \begin{tikzcd}
      \Omega(A,a)
      \arrow[r, "{\Omega(f,p_f)}"]
      \arrow[d,"\sim" sloped]
      & \Omega(B,b)
      \arrow[d,"\sim" sloped]
      \\
      \mathrm{PB}(a,a)
      \arrow[r]
      &
      \mathrm{PB}(b,b)
    \end{tikzcd}
    .
  \end{center}
\end{exercise}

\end{document}
