\documentclass{uebung}

\begin{document}
\maketitle{8}

\begin{exercise}[Äquivalente Basen induzieren äquivalente Summen]
  Sei $f:A \to B$ eine Funktion und $C : B \to \mU$ ein abhängiger Typ.
  \begin{enumerate}
    \item Finde eine kanonische Abbildung $\sum_{x:A} C(f(x)) \to \sum_{y:B} C(y)$.
    \item Sei $p:a=_A a'$. 
      Zeige $\transp_{C\circ f}(p) = \transp_C (f(p))$.
    \item Sei $f$ eine Äquivalenz.
      Zeige $\sum_{x:A} C(f(x)) \simeq \sum_{y:B} C(y)$.
  \end{enumerate}
\end{exercise}

\begin{exercise}[Gleichheit punktierter Typen]
  Ein \emph{punktierter Typ} ist ein Paar bestehend aus einem Typen $A$ und einem Term $a:A$.
  Den Typ $\sum_{A:\mU}A$ aller punktierten Typen in einem Universum $\mU$ bezeichnen wir mit $\mU_*$.

  Eine \emph{punktierte Äquivalenz} zwischen zwei punktierten Typen $(A,a)$, $(B,b)$ ist eine Äquivalenz $f:A\simeq B$ mit $f(a)=b$.
  Den Typ aller punktierten Äquivalenzen zwischen $(A,a)$ und $(B,b)$ definieren wir als
  $$
  (A,a) \simeq_* (B,b) \colonequiv \sum_{f:A\simeq B} f(a) = b.
  $$

  Finde für je zwei punktierte Typen $(A,a), (B,b) : \mU_*$ eine Äquivalenz
  $$
  \left((A,a) =_{\mU_*} (B,b)\right) \simeq \left((A,a) \simeq_* (B,b)\right).
  $$
\end{exercise}

\begin{exercise}
  Sei $A$ ein Typ.
  Zeige, dass $(S^1\to A) \simeq \sum_{x:A} x=x$.
\end{exercise}

\end{document}
