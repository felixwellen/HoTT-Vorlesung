\documentclass{uebung}

\DeclareMathOperator*{\whisk}{\ast}

\begin{document}
\maketitle{12}

\begin{exercise}[Pullbacks von Äquivalenzen sind Äquivalenzen]
  Betrachte ein Pullback-Quadrat der folgenden Form:
  \begin{center}
    \begin{tikzcd}
      X
      \arrow[r,"\varphi"]
      \arrow[d,"\psi"']
      & A
      \arrow[d,"f"]
      \arrow[dl, phantom, "H" font=\small]
      \\
      Y
      \arrow[r,"g"']
      & B
    \end{tikzcd}
  \end{center}
  \begin{enumerate}
    \item Sei $Y\equiv\einheit$.
      Zeige, dass $X\simeq\mathrm{fib}_f(g(\ast))$.
    \item Sei $f$ eine Äquivalenz.
      Zeige, dass $\psi$ eine Äquivalenz ist.
      {\tiny Tipp: Pullback-Pasting!}
  \end{enumerate}
\end{exercise}

\begin{exercise}[Funktorialität der Abschneidung]
  Sei $n:\N_{-2}$ ein Abschneidungslevel und $A:\mU$ ein Typ.
  \begin{enumerate}
    \item Konstruiere das Rekursionsprinzip für für die $n$-Abschneidung $\|A\|_n$ , finde also einen Term
      $$
      \rec{\|A\|_n}:\prod_{B:\nType{n}}(A \to B) \to (\|A\|_n \to B),
      $$
      sodass $\rec{\|A\|_n}(B,f)(|a|_n) \equiv f(a)$ für jedes $a:A$.
    \item Konstruiere das Eindeutigkeitsprinzip für für die $n$-Abschneidung $\|A\|_n$:
      $$
      \mathrm{uniq}_{\|A\|_n}:\prod_{B:\nType{n}}\prod_{g,g':\|A\|_n \to B} \left((g\circ |\_|_n) \sim (g'\circ |\_|_n)\right) \to g \sim g'
      $$
    \item
  Für $f:A\to B$ definieren wir $\|f\|_n\colonequiv \rec{\|A\|_n}(\|B\|_n,|\_|_n\circ f):\|A\|_n\to\|B\|_n$.
  Zeige:
  \begin{enumerate}
    \item $\|\id_A\|_n\sim \id_{\|A\|_n}$
    \item $\|g\|_n\circ\|f\|_n\sim\|g \circ f\|_n$.
  \end{enumerate}
  \end{enumerate}
\end{exercise}


\begin{exercise}[Universelle Eigenschaft der Abschneidung]
  Sei $n:\N_{-2}$ ein Abschneidungslevel, $A:\mU$, $B:\nType{n}$ und $f:A\to B$.
  Wir definieren
  \begin{align*}
    &f^*:\prod_{C:\nType{n}}(B \to C) \to (A \to C),\\
    &f^*(g)\colonequiv f^*(C,g)\colonequiv g\circ f.
  \end{align*}
  \begin{enumerate}
    \item Zeige, dass $\rec{\|A\|_n}(B,f):\|A\|_n\to B$ genau dann eine Äquivalenz ist, wenn $f^*$ eine faserweise Äquivalenz ist.
    %\item Folgere, dass Präkomposition mit $|\_|_n$ eine Äquivalenz $(\|A\|_n \to C) \to (A \to C)$ für jeden $n$-Typen $C:\nType{n}$ induziert.
    \item Folgere, dass $A$ genau dann ein $n$-Typ ist, wenn $|\_|_n:A\to \|A\|_n$ eine Äquivalenz ist.
    \item Folgere, dass $S^1$ zusammenhängend ist.
  \end{enumerate}
\end{exercise}

\begin{exercise}[Das Eckmann--Hilton Argument]
  Sei $A:\mU$ ein Typ und $a,b,c:A$.
  Wir erinnern an die Gruppoidgesetze für Komposition mit $\refl$
  \begin{align*}
    \rho:\prod_{p:a=b}p\kon\refl_b = p,\\
    \lambda:\prod_{r:b=c}\refl_b\kon r= r,
  \end{align*}
  welche sich per $=$-Induktion durch $\rho_{\refl_b} \colonequiv \lambda_{\refl_b} \colonequiv \refl_{\refl_b}$ definieren lassen.

  Wir definieren \emph{whiskering von links} als Operation vom Typ%
  \footnote{Die geschweiften Klammern deuten implizite Argumente an, die wir in der Notation der Operation nicht berücksichtigen.}
  $$
  \_\whisk_l\_:\prod_{\{a,b:A\}}\prod_{p:a=b}\prod_{\{c:A\}}\prod_{\{r,s:b=c\}}r=s\to p\kon r=p\kon s
  $$
  per $=$-Induktion über $p:a=b$ durch $\refl_b\whisk_l\beta \colonequiv\lambda_r\kon\beta\kon\lambda_s^{-1}$ für $\beta:r=s$.
  Analog definieren wir \emph{whiskering von rechts}
  $$
  \_\whisk_r\_:\prod_{\{a:A\}}\prod_{\{p,q:a=b\}}\prod_{\{b,c:A\}}\prod_{r:b=c}p=q\to p\kon r=q\kon r
  $$
  per $=$-Induktion über $r:b=c$ durch $\alpha\whisk_r\refl_b \colonequiv \rho_p\kon\alpha\kon\rho_q^{-1}$ für $\alpha:p=q$.

  Seien nun $a,b,c:A$, $p,q:a=b$, $r,s:b=c$ und $\alpha:p=q$, $\beta:r=s$ wie in folgendem Diagram veranschaulicht:
  \begin{center}
    \begin{tikzcd}[sep=large]
      a
        \arrow[r,equal,bend left,shift left=3,"p"{name=p}]
        \arrow[r,equal,bend right,shift right=3,"q"'{name=q}]
        \arrow[from=p,to=q,equal,shorten=6,"\alpha"]
      & b
        \arrow[r,equal,bend left,shift left=3,"r"{name=tikzsucks}]
        \arrow[r,equal,bend right,shift right=3,"s"'{name=s}]
        \arrow[from=tikzsucks,to=s,equal,shorten=6,"\beta"]
      & c
    \end{tikzcd}
  \end{center}
  Wir definieren die \emph{horizontale Konkatenation} von $\alpha$ und $\beta$ als den ersten der beiden Terme
  \begin{align*}
    \alpha\whisk\beta &\colonequiv (\alpha \whisk_r r) \cdot (q \whisk_l \beta),\\
    \alpha\whisk'\beta &\colonequiv (p \whisk_l \beta) \cdot (\alpha \whisk_r s).
  \end{align*}
  \begin{enumerate}
    \item Zeige, dass $\alpha\whisk\beta = \alpha\whisk'\beta$.
    \item Angenommen $a\equiv b\equiv c$ und $p\equiv q\equiv r\equiv s\equiv \refl_a$.
      Zeige, dass
      \begin{enumerate}
        \item $\alpha\whisk\beta = \alpha\kon\beta$,
        \item $\alpha\whisk'\beta = \beta\kon\alpha$.
      \end{enumerate}
    \item \textbf{Bonus.} Folgere, dass die Homotopiegruppe $\pi_k(A,a)$ für $k\geq 2$ abelsch ist.
  \end{enumerate}
\end{exercise}

\end{document}
