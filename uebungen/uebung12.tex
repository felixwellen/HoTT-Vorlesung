\documentclass{uebung}

\begin{document}
\maketitle{12}

\begin{exercise}[Pullbacks von Äquivalenzen sind Äquivalenzen]
  Betrachte ein Pullback-Quadrat der folgenden Form:
  \begin{center}
    \begin{tikzcd}
      X
      \arrow[r,"\varphi"]
      \arrow[d,"\psi"']
      & A
      \arrow[d,"f"]
      \arrow[dl, phantom, "H" font=\small]
      \\
      Y
      \arrow[r,"g"']
      & B
    \end{tikzcd}
  \end{center}
  \begin{enumerate}
    \item Sei $Y\equiv\einheit$.
      Zeige, dass $X\simeq\mathrm{fib}_f(g(\ast))$.
    \item Sei $f$ eine Äquivalenz.
      Zeige, dass $\psi$ eine Äquivalenz ist.
      {\tiny Tipp: Pullback-Pasting!}
  \end{enumerate}
\end{exercise}

\begin{exercise}[Funktorialität der Abschneidung]
  Sei $n:\N_{-2}$ ein Abschneidungslevel und $A:\mU$ ein Typ.
  \begin{enumerate}
    \item Konstruiere das Rekursionsprinzip für für die $n$-Abschneidung $\|A\|_n$ , finde also einen Term
      $$
      \rec{\|A\|_n}:\prod_{B:\nType{n}}(A \to B) \to (\|A\|_n \to B),
      $$
      sodass $\rec{\|A\|_n}(B,f)(|a|_n) \equiv f(a)$ für jedes $a:A$.
    \item Konstruiere das Eindeutigkeitsprinzip für für die $n$-Abschneidung $\|A\|_n$:
      $$
      \mathrm{uniq}_{\|A\|_n}:\prod_{B:\nType{n}}\prod_{g,g':\|A\|_n \to B} \left((g\circ |\_|_n) \sim (g'\circ |\_|_n)\right) \to g \sim g'
      $$
    \item
  Für $f:A\to B$ definieren wir $\|f\|_n\colonequiv \rec{\|A\|_n}(\|B\|_n,|\_|_n\circ f):\|A\|_n\to\|B\|_n$.
  Zeige:
  \begin{enumerate}
    \item $\|\id_A\|_n\sim \id_{\|A\|_n}$
    \item $\|g\|_n\circ\|f\|_n\sim\|g \circ f\|_n$.
  \end{enumerate}
  \end{enumerate}
\end{exercise}


\begin{exercise}[Universelle Eigenschaft der Abschneidung]
  Sei $n:\N_{-2}$ ein Abschneidungslevel, $A:\mU$, $B:\nType{n}$ und $f:A\to B$.
  Wir definieren
  \begin{align*}
    &f^*:\prod_{C:\nType{n}}(B \to C) \to (A \to C),\\
    &f^*(g)\colonequiv f^*(C,g)\colonequiv g\circ f.
  \end{align*}
  \begin{enumerate}
    \item Zeige, dass $\rec{\|A\|_n}(B,f):\|A\|_n\to B$ genau dann eine Äquivalenz ist, wenn $f^*$ eine faserweise Äquivalenz ist.
    %\item Folgere, dass Präkomposition mit $|\_|_n$ eine Äquivalenz $(\|A\|_n \to C) \to (A \to C)$ für jeden $n$-Typen $C:\nType{n}$ induziert.
    \item Folgere, dass $A$ genau dann ein $n$-Typ ist, wenn $|\_|_n:A\to \|A\|_n$ eine Äquivalenz ist.
    \item Folgere, dass $S^1$ zusammenhängend ist.
  \end{enumerate}
\end{exercise}

\begin{exercise}[Eckmann-Hilton und H-Gruppen]
  Sei $A:\mU$ ein Typ, $a,b,c:A$.
  Für $p:a=b$ definieren wir \emph{whiskering von links}
  \begin{align*}
    \prod_{p:a=b}\prod_{r,s:b=c}r=s\to p\kon r=p\kon s\\
  \end{align*}
\end{exercise}

\end{document}
