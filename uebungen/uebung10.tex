\documentclass{uebung}

\begin{document}
\maketitle{10}

\begin{exercise}[Retrakte von $n$-Typen]
  Sei $r:A \to B$ eine \emph{Retraktion}, existiere also eine Rechtsinverse $s:B \to A$ zu $r$, und sei $A$ ein $n$-Typ.
  Zeige, dass $B$ ein $n$-Typ ist.
\end{exercise}

\begin{exercise}[Koprodukte von kontrahierbaren Typen und Aussagen]
  Seien $A,B:\mU$.
  Zeige:
  \begin{enumerate}
    \item Sind $A$ und $B$ kontrahierbar, so ist $A \amalg B$ nicht kontrahierbar.
    \item Sind $A$ und $B$ Aussagen, so ist $A \amalg B$ eine Aussage genau dann, wenn $A \to \neg B$.
  \end{enumerate}
\end{exercise}

\begin{exercise}[Diese anderen ganzen Zahlen]
  Sei $\Z'$ der induktiv definierte Typ der ganzen Zahlen aus Abschnitt 3.3.
  Finde eine Funktion $\N\times \N\to\Z'$, welche eine Äquivalenz $\Z\to\Z'$ induziert.
\end{exercise}

\end{document}
