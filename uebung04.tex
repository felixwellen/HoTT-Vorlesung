\documentclass{uebung}

\begin{document}
\maketitle{4}

\begin{exercise}[Zwei Isomorphismen]
  Seien $A,B,C$ Typen.
  Konstruiere zueinander inverse Funktionen zwischen den folgenden Typen:
  \begin{enumerate}
    \item $A$ und $\einheit$, falls $A$ kontrahierbar ist.
    \item $(A \amalg B) \to C$ und $(A\to C) \times (B\to C)$.
      {\tiny Eine Richtung benötigt Funktionsextensionalität.}
  \end{enumerate}
\end{exercise}

\begin{exercise}[Inverse sind eindeutig]
  Seien $A,B,C$ Typen.
  \begin{enumerate}
  \item Seien $f,f':A\to B$ Funktionen und $f\sim f'$, also $f$ homotop zu $f'$.
    Dann ist für $h:B\to C$, auch $h\circ f$ homotop zu $h\circ f'$.
    Analog ist auch für $i:Z\to A$ die Funktion $f\circ i$ homotop zu $f'\circ i$.
  \item Seien $f:A\to B$ und $g:B\to A$, $g':B\to A$ jeweils eine Invers von $f$.
    Dann sind $g$ und $g'$ homotop.
  \item Seien $f:A\to B$ und $g:B\to A$ invers zueinander und $f':B\to C$, $g':C\to B$ invers zueinander.
    Dann sind auch $f'\circ f$ und $g\circ g'$ invers zueinander.
  \end{enumerate}
\end{exercise}

\begin{exercise}[Speziellere Rechengesetze für Gleichheiten]
  Seien $A$ ein Typ und $x,y,z:A$. Zeige:
  \begin{enumerate}
  \item Für Gleichheiten $p:x=y$ gilt $\left(p^{-1}\right)^{-1}=p$.
  \item Für Gleichheiten $p:x=y$ und $q:y=z$ gilt: $(p\kon q)^{-1}=q^{-1}\kon p^{-1}$.
  \end{enumerate}
\end{exercise}

\begin{bonus}[Kontrahierbarkeit ist eine Aussage]
  Zeige, dass für jeden Typ $A$ der Typ
  \begin{mathpar}
    \isContr(A)\equiv\sum_{c:A}\prod_{x:A}x=c
  \end{mathpar}
  ein Aussage ist.
\end{bonus}

\end{document}
