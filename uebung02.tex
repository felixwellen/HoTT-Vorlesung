\documentclass{uebung}

\begin{document}
\maketitle{2}

\begin{exercise}[Zwei Einführungsregeln für Nachfolger]
  Zeige, dass die folgenden beiden Einführungsregeln für den Nachfolger einer natürlichen Zahl äquivalent sind:
  \begin{mathpar}
    \inferrule{
      \Gamma\vdash n:\N
    }{
      \Gamma\vdash\sucN(n):\N
    }{
      \N\mathrm{I2}
    }
    \and
    \inferrule{
      \Gamma\text{ Kontext}
    }{
      \Gamma,x:\N\vdash\sucN(x):\N
    }{
      \N\mathrm{I2'}
    }
  \end{mathpar}
  {\tiny Die linke Regel ist aus dem Skript, die rechte aus der Vorlesung.}
\end{exercise}

\end{document}
