Es gibt viele unterschiedliche Typentheorien.
Wir beschäftigen uns hier hauptsächlich mit der Variante CCHM, zu der man im Artikel ``Cubical Type Theory: A constructive interpretation of the univalence axiom'' von Cohen, Coquand, Huber und Mörtberg mehr nachlesen kann.
Diese Typentheorie ist recht nah an den kubischen Elementen der Programmiersprache Agda.

Kubische Typentheorien (cubical type theories) wurden entwickelt,
um die typentheoretischen Mängel der Homotopietypentheorie zu beheben.
Zentral ist dabei die Idee, Gleichheiten durch Abbildungen zu modellieren.
Allerdings ist das Interval, die Quelle dieser Abbildunge typischerweise kein Typ und daher sind Typen von Gleichheiten auch nicht die Funktionstypen, die wir bisher gesehen haben.

